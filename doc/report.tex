\documentclass[a4paper,12pt,titlepage,finall]{article}

\usepackage{hyperref}
\usepackage{amssymb}
\usepackage[T1,T2A]{fontenc}     % форматы шрифтов
\usepackage[utf8x]{inputenc}     % кодировка символов, используемая в данном файле
\usepackage[russian]{babel}      % пакет русификации
\usepackage{tikz}                % для создания иллюстраций
\usepackage{pgfplots}            % для вывода графиков функций
\usepackage{geometry}		 % для настройки размера полей
\usepackage{indentfirst}         % для отступа в первом абзаце секции
\setlength{\abovedisplayskip}{5mm}
\setlength{\belowdisplayskip}{5mm}
% выбираем размер листа А4, все поля ставим по 3см
\geometry{a4paper,left=30mm,top=30mm,bottom=30mm,right=30mm}

\setcounter{secnumdepth}{0}      % отключаем нумерацию секций

\usepgfplotslibrary{fillbetween} % для изображения областей на графиках

% Настройка отображения графов
\usetikzlibrary{shapes,arrows,positioning}
\tikzset{
    block/.style = {rectangle, draw, text width=3.5cm, text centered, rounded corners, minimum height=1cm},
    line/.style = {draw, -latex'},
    cloud/.style = {draw, ellipse, minimum height=1cm},
}

\begin{document}
% Титульный лист
\begin{titlepage}
    \begin{center}
	{\small \sc Московский государственный университет \\имени М.~В.~Ломоносова\\
	Факультет вычислительной математики и кибернетики\\}
	\vfill
	{\Large \sc Отчет по заданию №6}\\
	~\\
	{\large \bf <<Сборка многомодульных программ. \\
	Вычисление корней уравнений и определенных интегралов.>>}\\ 
	~\\
	{\large \bf Вариант 11 / 1-4 / 1-3}
    \end{center}
    \begin{flushright}
	\vfill {Выполнил:\\
	студент 102 группы\\
	Такшин~А.~И.\\
	~\\
	Преподаватель:\\
	Гуляев~Д.~А.}
    \end{flushright}
    \begin{center}
	\vfill
	{\small Москва\\2024}
    \end{center}
\end{titlepage}

% Автоматически генерируем оглавление на отдельной странице
\tableofcontents
\newpage

\section{Постановка задачи}
\begin{itemize}
  \item Требуется реализовать программу, которая численными методами вычисляет
    площадь фигуры, ограниченной тремя кривыми заданными ввиде формул 
    $f:\mathbb{R} \rightarrow \mathbb{R}$.
\item Для вычисления площади реализованны численные методы интегрирования: через формулу прямоугольников, через формулу трапеций и, используемый по умолчанию, через формулу Симпсона.
\item Для нахождения вершин криволинейного треугольника реализованы 4 метода нахождения корня функции на заданном отрезке: метод деления отрезка пополам, метод хорд, метод Ньютона и комбинированный метод (метод хорд + метод Ньютона).
\item Так как для корректной работы перечисленных методов необходимо, чтобы значения функции на концах имели разные знаки, отрезок для их применения должен быть вычислен заранее аналитически и задан в файле с описанием функций.
\end{itemize}

\newpage

\section{Математическое обоснование}
\subsection{Алгоритм нахождения площади криволинейного треугольника}
Рассмотрим на примере (рис.~\ref{plot1}) порядок вычисления площади криволинейного треугольника образованного графиками трёх функций, в предположении, что любые из двух функций $f_1,f_2,f_3$ пересекаются ровно в одной точке на заданном отрезке.

Упорядочим точки попарных пересечений функций $\{A,B,C\}$, тогда для вычисления
площади достаточно сложить модули интегралов функций $g_1(x),g_2(x)$ на отрезках 
$[A_x;B_x]$, $[B_x;C_x]$, где $g_1(x)$ разность функций, точка пересечения которых 
является точкой $A$, а $g_2(x)$ разность функций, точка пересечения которых 
является точкой $C$. Так как функции пересекаются в одной точке, знак разности любых двух из них не будет меняться после прохождения точки пересечения, а значит полученная сумма будет корректно определять площадь.

В данном примере, точка $A$ является точкой пресечения $f_2(x),f_3(x)$, поэтому
программа изначально вычислит модуль интеграла на отрезке $[A_x,B_x]$ функции 
$g_1(x) = f_2(x) - f_3(x)$. Точка $C$ является точкой пересечения $f_1(x), f_3(x)$,
поэтому вторым слагаемым будет модуль интеграла $g_2(x) = f_1(x) - f_2(x)$

$S(\Phi) = |\int_{A_x}^{B_x} f_2(x) - f_3(x)\,dx| + |\int_{B_x}^{C_x} f_1(x) - f_3(x)\,dx|$

\begin{figure}[h]
\centering
\begin{tikzpicture}
\begin{axis}[% grid=both,                % рисуем координатную сетку (если нужно)
             axis lines=middle,          % рисуем оси координат в привычном для математика месте
             restrict x to domain=-2:4,  % задаем диапазон значений переменной x
             restrict y to domain=-1:6,  % задаем диапазон значений функции y(x)
             axis equal,                 % требуем соблюдения пропорций по осям x и y
             enlargelimits,              % разрешаем при необходимости увеличивать диапазоны переменных
             legend cell align=left,     % задаем выравнивание в рамке обозначений
             scale=2]                    % задаем масштаб 2:1

% первая функция
% параметр samples отвечает за качество прорисовки
\addplot[green,domain=-2:4,samples=256,thick] {(2 - tan(deg(x)/4))};
% описание первой функции
\addlegendentry{$f_1(x)=2-tan(\frac{x}{4})$}

% добавим немного пустого места между описанием первой и второй функций
\addlegendimage{empty legend}\addlegendentry{}

% вторая функция
% здесь необходимо дополнительно ограничить диапазон значений переменной x
\addplot[blue,domain=-0.5:4,samples=256,thick] {x};
\addlegendentry{$f_2(x)=x$}

% дополнительное пустое место не требуется, так как формулы имеют небольшой размер по высоте

% третья функция
\addplot[red,samples=256,thick] {0.2*pi};
\addlegendentry{$f_3(x)=0.2\cdot\pi$}
\end{axis}
\end{tikzpicture}
\caption{Пример плоской фигуры, ограниченной графиками заданных уравнений}
\label{plot1}
\end{figure}

\subsection{Формулы оценки погрешностей методов}
Абсолютная погрешность по абциссе при вычислении точки пересечения $\varepsilon_1$
задется непосредственно в вычисляющем её методе. 
Для нахождения значения $\varepsilon_2$ - абсолютной погрешности при вычислении
определенного интеграла
использовались известные формулы оценки погрешности ~\cite{math}:
\begin{itemize}

  \item $\varepsilon_2 = \frac{f'(\xi)}{2} h^2 (b-a)$, для формулы прямоугольников

  \item $\varepsilon_2 = \frac{f''(\xi)}{12} h^2 (b-a)$, для формулы трапеций

  \item $\varepsilon_2 = \frac{f^{(4)}(\xi)}{2880} h^4 (b-a)$, для формулы Симпсона
\end{itemize}
Учитывая, что $h = \frac{(b-a)}{n}$ и предполагая $f^{(n)} \approx 1$, получаем следующие
оценки на число шагов:

\begin{itemize}
  \item $ n = \sqrt[2]{\frac{(b-a)^3}{2\varepsilon_2}}$, для формулы прямоугольников

  \item $ n = \sqrt[2]{\frac{(b-a)^3}{12\varepsilon_2}}$, для формулы трапеций

  \item $ n = \sqrt[4]{\frac{(b-a)^5}{2880\varepsilon_2}}$, для формулы Симпсона
\end{itemize}

\subsection{Оценка общей погрешности}
Пусть с помощью описанных выше методом мы получили оценку $I'$ для интеграла
на отрезке $[a';b']$, где 
$a' = a + \varepsilon_1,~b'=b+\varepsilon_1$
, и пусть $I$ - действительное значение интеграла на отрезке $[a;b]$. 
Тогда из разложения в ряд Тейлора:
\[I' = I + f(a)\varepsilon_1 + f(b)\varepsilon_1 + o(\varepsilon_1)\]
\[I' - I \approx f(a)\varepsilon_1 + f(b)\varepsilon_1\]
Итоговая ошибка вычисления двух интегралов составит:
\[\varepsilon_3 = (f(A_x) + 2f(B_x) + f(C_x))\varepsilon_1\]

Разобьём требуемую наибольшую ошибку $\varepsilon$ пополам между $\varepsilon_3$ и $\varepsilon_2$. Тогда итоговые оценки для $\varepsilon_1$ и $\varepsilon_2$ будут такими:
\begin{itemize}
  \item $\varepsilon_1 = \frac{\varepsilon}{2}$

  \item $\varepsilon_2 = \frac{\varepsilon}{2(f(A_x) + 2f(B_x) + f(C_x)}$
\end{itemize}
\newpage

\section{Результаты экспериментов}

В результате работы программы для тестового примера (рис.~\ref{plot1}) были получены следующие координаты точек пересечения:

\begin{table}[h]
\centering
\begin{tabular}{|c|c|c|}
\hline
Кривые & $x$ & $y$ \\
\hline
1 и 2 &  1.5824 & 1.5824 \\
2 и 3 &  0.6283 & 0.6283 \\
1 и 3 &  3.7634 & 0.6283 \\
\hline
\end{tabular}
\caption{Координаты точек пересечения}
\label{table1}
\end{table}


После чего программа корректно нашла искомую площадь криволинейного прямоугольника (рис.~\ref{plot2}).

\begin{figure}[h]
\centering
\begin{tikzpicture}
\begin{axis}[% grid=both,                % рисуем координатную сетку (если нужно)
             axis lines=middle,          % рисуем оси координат в привычном для математики месте
             restrict x to domain=-2:4,  % задаем диапазон значений переменной x
             restrict y to domain=-1:6,  % задаем диапазон значений функции y(x)
             axis equal,                 % требуем соблюдения пропорций по осям x и y
             enlargelimits,              % разрешаем при необходимости увеличивать диапазоны переменных
             legend cell align=left,     % задаем выравнивание в рамке обозначений
             scale=2,                    % задаем масштаб 2:1
             xticklabels={,,},           % убираем нумерацию с оси x
             yticklabels={,,}]           % убираем нумерацию с оси y

% первая функция
% параметр samples отвечает за качество прорисовки
\addplot[green,samples=256,thick,name path=A] {(2 - tan(deg(x)/4))};
% описание первой функции
\addlegendentry{$f_1(x)=2-tan(\frac{x}{4})$}

% добавим немного пустого места между описанием первой и второй функций
\addlegendimage{empty legend}\addlegendentry{}

% вторая функция
% здесь необходимо дополнительно ограничить диапазон значений переменной x
\addplot[blue,domain=-0.5:4,samples=256,thick,name path=B] {x};
\addlegendentry{$f_2(x)=x$}

% дополнительное пустое место не требуется, так как формулы имеют небольшой размер по высоте

% третья функция
\addplot[red,samples=256,thick,name path=C] {0.2*pi};
\addlegendentry{$f_3(x)=0.2\cdot\pi$}

% закрашиваем фигуру
\addplot[blue!20,samples=256] fill between[of=C and B,soft clip={domain=0.6283:1.5824}];
\addplot[blue!20,samples=256] fill between[of=A and C,soft clip={domain=3.7634:1.5824}];
\addlegendentry{$S=1.6514$}

% Поскольку автоматическое вычисление точек пересечения кривых в TiKZ реализовать сложно,
% будем явно задавать координаты.
\addplot[dashed] coordinates { (3.7634, 0.6283) (3.7634, 0) };
\addplot[color=black] coordinates {(3.7634, 0)} node [label={-45:{\small 3.7634}}]{};

\addplot[dashed] coordinates { (0.6283, 0.6283) (0.6283, 0) };
\addplot[color=black] coordinates {(0.6283, 0)} node [label={-45:{\small 0.6283}}]{};

\addplot[dashed] coordinates { (1.5824, 1.5824) (1.5824, 0) };
\addplot[color=black] coordinates {(1.5824, 0)} node [label={-45:{\small 1.5824}}]{};

\end{axis}
\end{tikzpicture}
\caption{Плоская фигура, ограниченная графиками заданных уравнений}
\label{plot2}
\end{figure}

\newpage

\section{Структура программы и спецификация функций}

Программа состоит из следующих файлов

\begin{itemize}
  \item[] Основные файлы:
    \begin{itemize}
      \item[$\bullet$] main.c - устанавливает порядок запуска функций из других файлов, обрабатывает аргументы командной строки.
      \item[$\bullet$] root.c/h - содержит функции, вычисляющие пересечение заданной функции с осью абцисс.
      \item[$\bullet$] integral.c/h - содержит функции, вычисляющие интеграл
    \end{itemize}
  \item[] Файлы подпрограммы, которая переводит польскую нотацию в asm код:
  \begin{itemize}
    \item[$\bullet$] function\_to\_asm.c - считывает из заданного файла польскую нотацию, вызывает функции из других файлов. На выходе возваращет в stdin полученный листинг ассемблера.
    \item[$\bullet$] operation\_tree.c/h - строит дерево, описывающие порядок вычисления выражений. Разрешает проблемы связанные с нехваткой регистров x87 процессора.
    \item[$\bullet$] constant.h - содержит константы, необходимые для работы описанных выше функций.
\end{itemize}
  \item[] Файлы вспомогательных функций 
    \begin{itemize}
      \item[$\bullet$] cvector.c/h - содержит реализацию динамического массива.
      \item[$\bullet$] tools.c/h - содержит вспомогательные простые функции.
      \item[$\bullet$] macro.h - содержит макросы препроцессора.
    \end{itemize}
\end{itemize}
\begin{figure}[h]
\begin{tikzpicture}[node distance = 1.5cm, auto]
    % MAIN
    \node [block] (main) {main.c};
    \node [block, below of=main, xshift=-5cm, text width=2.5cm] (root) {root.c/h};
    \node [block, below of=main, text width=3cm] (integral) {integral.c/h};
    \node [block, below of=main,text width=2.5cm, xshift=5cm] (segment) {segment.h};
    % COMPILER
    \node [cloud, above of=main] (asm) {func.asm};
    \node [block, above of=asm, xshift=3cm] (func) {function\_to\_asm.c};
    \node [block, above of=asm, xshift=-1.5cm] (op_tree) {operation\_tree.c/h};

    % LIB
    \node [block, above of=asm, yshift=1.5cm, xshift=-3.5cm] (macro) {macro.h};
    \node [block, below of=func, xshift=3cm] (constants) {constants.h};
    \node [block, above of=macro, xshift=5cm] (tools) {tools.c/h};
    \node [block, above of=macro] (cvector) {cvector.c/h};

    % Draw edges
    \path [line] (main) -- (root);
    \path [line] (main) -- (integral);
    \path [line] (main) -- (asm);
    \path [line] (main) -- (segment);
    \path [line] (root) -- (macro);
    \path [line] (func) -- (asm);
    \path [line] (func) -- (segment);
    \path [line] (func) -- (op_tree);
    \path [line] (func) -- (macro);
    \path [line] (func) -- (cvector);
    \path [line] (func) -- (tools);
    \path [line] (func) -- (constants);
    \path [line] (op_tree) -- (macro);
    \path [line] (cvector) -- (macro);
    \path [line] (tools) -- (macro);

\end{tikzpicture}
\caption{Графическое представление структуры программы}
\label{graph1}
\end{figure}
\newpage

\section{Сборка программы (Make-файл)}

В данном разделе необходимо описать зависимости между модулями программы
и привести текст Make-файла. Зависимости проще всего описать диаграммой.

\newpage

\section{Отладка программы, тестирование функций}
Для тестирования функций и отладки программы использовался специальный ключ препроцессора (-DDEBUG).

Функции описанные в файле root.c были протестированы на следующих примерах: $f(x) = x,~f(x) = x^2 - 2, ~f(x) = cos(x)$, для каждого из тестов подбирался соответсвующий отрезок для поиска корня: $[-1; 1],~[0; 2],~[1;4]$. Полученные результаты полностью совпали с вычисленными аналитически.

Функции из файла integral.c тестировались сразу в полной сборке программы. Вот входных данных в spec.txt, задающем кривые ограничивающие криволинейный треугольник, аналитически вычисленные значения площади и полученые результаты в ходе работы программы:

\begin{table}[h]
\centering
\begin{tabular}{|c|c|c|c|}
\hline
Curves & Segment & Predicted & Stdout \\
\hline
$y_1 = 0,~y_2=x,~y_3=2-x$ & $[-4;4]$ &  1.0000 & 1.0000 \\
$y_1 = 2-tg(\frac{x}{4}),~y_2=x,~y_3=0.2\cdot \pi$ & $[0;4]$ &  1.6514 & 1.6516 \\
$y_1 = cos(x),~y_2 = 2x, ~y_3 =0$ &$[-1;2] & 0.7900 & 0.7678 \\
\hline
\end{tabular}
\caption{Результаты тестирования программы}
\label{table2}
\end{table}

В результате тестирования программы, было обнаружено, 
что на некоторых данных погрешность получается больше ожидаемой.
Данная проблема не исправляется даже при уменьшении $\varepsilon \leq 10^{-7}.$

Тем не менее, результаты программы в большинстве случаев сходятся с прогрнозируемыми с требуемой точностью. 
\newpage

\section{Программа на Си и на Ассемблере}
Исходные тексты программ имеются в архиве, который приложен к этому отчету. 
А ещё исходные тексты программ можно найти на \href{https://github.com/SadWork/msu_task02}{репозитории github}.
\newpage

\section{Анализ допущенных ошибок}

\newpage
\begin{raggedright}
\addcontentsline{toc}{section}{Список цитируемой литературы}
\begin{thebibliography}{99}
\bibitem{math} Ильин~В.\,А., Садовничий~В.\,А., Сендов~Бл.\,Х. Математический анализ. Т.\,1~---
    Москва: Наука, 1985.
\end{thebibliography}
\end{raggedright}

\newpage


\end{document}
